
\documentclass{article}
\usepackage[margin=1in]{geometry}
\usepackage{amsmath, amssymb, amsthm}
\usepackage{indentfirst}

% Custom Commands
\newcommand{\N}{\mathbb{N}}

% Metadata
\title{Math Homework}
\author{John Doe}
\date{12/23/2025}

\begin{document}
\maketitle


\section*{Problem 1}




\subsection*{a)}

Proof:

\begin{itemize}
\item[>] let $a = 0$ and arbitrary $b \in \mathbb{N}$
\item[>] 
\item[>] $0 \in \mathbb{N}$, so $a \in \mathbb{N}$
\item[>] 
\item[>] By definition of natural numbers, $0$ is the smallest natural number. Thus $a = 0$ is the smallest natural number. Thus the statement $a \leq b$ holds true, since arbitrary $b$ can only be $\geq 0$
\item[>] 
\item[>] Thus $a \leq b$ when $a = 0$, satisfying the condition $\blacksquare$
\end{itemize}




\subsection*{b)}

$x^2 - 2x - 8 = 0, (x-4)(x+2) = 0, x = 4, -2$

Roots Of Quadratic Equation: $4, -2$






\end{document}