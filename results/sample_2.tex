
\documentclass{article}
\usepackage[margin=1in]{geometry}
\usepackage{amsmath, amssymb, amsthm}
\usepackage{indentfirst}

% Custom Commands
\newcommand{\N}{\mathbb{N}}

% Metadata
\title{Math Homework}
\author{John Doe}
\date{12/23/2025}

\begin{document}
\maketitle


\section*{Problem 1}


Proof:

\begin{itemize}
\item[>] Assume there exists an $a \in \mathbb{N}$ such that $\forall a \in \mathbb{N}$ for all $b \in \mathbb{N}$.
\end{itemize}

\begin{itemize}
\item[>] $a - 1 \in \mathbb{N}$, so it is not true that $a \in \mathbb{N}$ for all $b \in \mathbb{N}$. Therefore, the assumption is false. $\blacksquare$
\end{itemize}




\section*{Problem 2}




\subsection*{a)}

$2 + 4 = 6$




\subsection*{b)}

$40 \% 3 = 1$





\section*{Problem 3}




\subsection*{a)}




\subsubsection*{i)}
The set of states is $\mathbb{N} \times \mathbb{N} : \{(h,t) \mid h,t \in \mathbb{N}^{\geq 2}\}$, where $(h,t)$ represents the state with $h$ heads and $t$ tails.

\subsubsection*{ii)}
The start state is $(40, 4)$.



\subsection*{b)}

Proof:

\begin{itemize}
\item[>] $P(n) := a_n \leq c \cdot 2^n$
\end{itemize}

\begin{itemize}
\item[>] This is a sample proof! $\blacksquare$
\end{itemize}






\end{document}